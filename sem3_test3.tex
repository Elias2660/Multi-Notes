\documentclass{article}
\usepackage{amsmath}
\usepackage{hyperref}
\usepackage{amsfonts}
\usepackage{bookmark}
\usepackage{float} 
\usepackage{graphicx}
\usepackage{makeidx}
\usepackage{pgfplots}
\usepackage[letterpaper, total={7.5in, 10in}]{geometry}


\makeindex
\pgfplotsset{compat=1.18}
\newcommand{\CURL}{\operatorname{curl}}
\newcommand{\DIV}{\operatorname{div}}
\begin{document}
\title{Test 3 Prep \\
    \large{Semester 2}}
\author{Elias Xu}
\date{\today}
\maketitle

\tableofcontents

\setlength{\parindent}{0pt}

\section{Introduction}

The topics for the test are:

\begin{itemize}
    \item Refresher on Spherical and Cylindrical coordinates
    \item Refresher on Double and Triple Integrals
    \item The Jacobian
    \item Line Integrals
    \item FTC for line integrals
    \item Vector Fields
    \item Line integrals of Vector functions
    \item Conservative Vector Fields (esp. Tests for this stuff)
    \item Green's Theorem
    \item DIV and CURL
\end{itemize}

\section{The Jacobian and Converting Between Coordinate Planes}

For double integrals:

\[
    \iint_R f(x, y) \delta x \delta y = \iint_S f(x(u, v), y(u,v)) \frac{\delta (x, y)}{\delta (u, v)} \delta u \delta v
\]
\[
    \frac{\delta (x, y)}{ \delta (u, v)} = \begin{vmatrix}
        \frac{\delta x}{\delta u} & \frac{\delta x}{\delta v}  \\
        \frac{\delta y}{\delta u} & \frac{\delta y}{ \delta v}
    \end{vmatrix}
\]

3 Variable Jacobian:

\[
    \frac{\delta (x, y, z)}{\delta (u, v, w)} = \begin{vmatrix}
        \frac{\delta x}{ \delta u } & \frac{\delta x}{\delta v} & \frac{\delta x}{\delta w} \\
        \frac{\delta y}{\delta u}   & \frac{\delta y}{\delta v} & \frac{\delta y}{\delta w} \\
        \frac{\delta z}{\delta u}   & \frac{\delta z}{\delta v} & \frac{\delta z}{\delta w}
    \end{vmatrix}
\]

\section{Line Integrals}

The area that can be taken under the cross section of a surface by a path.

\subsection{W/R to dt}

$$\int_C f(x, y) \delta s$$

$\delta s$ represents a small arc length, and thus can be represented through a parameterization (in this example given a 2D function)

$$\delta s = \sqrt{(x'(t))^2 + (y'(t))^2} \delta t$$

thus

$$\int_{t_0}^{t_1} f(x, y) \delta s = \int_{t_0}^{t_1} f(x, y) \sqrt{(x'(t)) ^2 + (y'(t))^2 + (z'(t))^2} \delta t$$

\subsection{W/R to dx and dy}

Projecting a path on the x or y axis rather than over the entire curve

\textbf{Theorem 1}: If $\vec{r}(t) = <x(t), y(t)>$ represents a smooth parameterization of C, then:

$$\int_C f(x, y) \delta x = \int_{t = a}^{t = b} f(x(t), y(t)) \cdot x'(t) \delta t$$
$$\int_C f(x, y) \delta y = \int_{t = a}^{t = b} f(x(t), y(t)) \cdot y'(t) \delta t$$


\section{Vector Fields}

A vector field is a function that can assign a vector to any point. Very similar but not completely equal to a gradient.

Calculating the "work" done by a vector field:

$$\int_c \vec{F} \vec{T} \delta s = \int_c \vec{F} \frac{\vec{r}(t)}{\|\vec{r'}(t)\|} \|\vec{ r}'(t) \| \delta t = \int_c \vec{F} \cdot \vec{r'}(t) \delta t$$


\subsection{Conservative Vector Fields}

Conservative vector fields are always path independent, additionally they are the gradient of a function. One can use Clairaut's to find it. 

\textbf{Theorem}: If $\vec{F}(x, y) = <P(x, y), Q(x, y)>$ is a vector field defined on an open connected set, D and if F is path independent in D then F is conservative in D.

That means that you can find a line integral by finding the original function and then plugging the beginning and end points. 

\subsection{Green's Theorem}

(Equivalent to FTC)

Let C be a piecewise, sooth, positively oriented simple closed curve and let D be the region enclosed by C and supposed that P(x, y) and Q(x, y have continuous) partials on an open region containing D. Then 

$$\int_C P(x, y) \delta x + Q(x, y) \delta y = \iint_D \frac{\delta Q}{\delta x} - \frac{\delta P}{\delta y} \delta A$$


\section{DIV and CURL}

CURL os a measure the tendency of particles to rotate at a point. Think of it as measuring the spin of a top. ($F(x, y, z) = <P(x, y, z), Q(x,y,z), R(x,y,z)$)

\[
\CURL(\vec{F}) = \vec{\nabla} \times \vec{F} = \begin{vmatrix}
    \hat{i} & \hat{j} & \hat{k} \\
    \delta x & \delta y & \delta z \\
    P & Q & R
\end{vmatrix}
\]

DIV is a difference of the flux leaving / entering a point. 

\[
\DIV(\vec{F}) =  \vec{\nabla} \cdot \vec{F} = \delta x P + \delta y Q + \delta z R
\]



For both, if conservative equal zero.


\end{document}